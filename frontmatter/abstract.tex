%
% File: abstract.tex
% Author: V?ctor Bre?a-Medina
% Description: Contains the text for thesis abstract
%
% UoB guidelines:
%
% Each copy must include an abstract or summary of the dissertation in not
% more than 300 words, on one side of A4, which should be single-spaced in a
% font size in the range 10 to 12. If the dissertation is in a language other
% than English, an abstract in that language and an abstract in English must
% be included.

\chapter*{Abstract}
\begin{SingleSpace}
\initial{O}ne of the biggest questions of astro-particle physics is the existence of dark matter. In recent years many experiments have been aiming at solving this question. Unfortunately to this date all measurements were null results, pushing the field to constantly try to extend the edge of its ability.
In this thesis three studies elated to different aspects motivated by  different phases in direct detection of dark matter experiments.

The first is a search for high energy nuclear recoil with the XENON100 detector based on models coming from effective field theory approach for dark matter-nucleus interactions including both elastic dark matter and inelastic dark matter. I present the first study on recoil energies above 40\,keV in the XENON100 detector, searching for signature of interactions coming from these models.

The second study is the develop of a liquid xenon apparatus aiming at measuring quantum scintillation properties of liquid xenon, mainly searching for supperradiance effects. A better understanding of these process can help in discriminating background currently considered as irreducible.

The third study is on an external calibration technique for XENON1T detector. In order to discriminate between background and the expected signal, intensive calibration is required. Ton-scale detectors cannot be calibrated using the techniques previously used due to the self shielding of the xenon. I present a technique used in XENON1T to obtain calibration of electromagnetic interactions. 



\end{SingleSpace}
\clearpage