%
% File: abstract.tex
% Author: V?ctor Bre?a-Medina
% Description: Contains the text for thesis abstract
%
% UoB guidelines:
%
% Each copy must include an abstract or summary of the dissertation in not
% more than 300 words, on one side of A4, which should be single-spaced in a
% font size in the range 10 to 12. If the dissertation is in a language other
% than English, an abstract in that language and an abstract in English must
% be included.

\chapter*{Abstract}
\begin{SingleSpace}
\initial{O}ne of the biggest open questions of astro-particle physics is the nature of dark matter. In recent years many experiments have been aiming at solving this question. Unfortunately to this date all measurements were null results, pushing the field to constantly try to extend the edge of its ability.
In this thesis I present three studies related to different aspects motivated by  different phases in direct detection of dark matter experiments. Each study aims at improving the sensitivity by: increasing energy range; increasing detector volume; decreasing background.

The first project is a search for high energy nuclear recoils with the \textsc{Xenon100} detector based on models coming from effective field theory approach for dark matter-nucleus interactions including both simple dark matter and inelastic dark matter. I present the first study on recoil energies above 40\,keV in the \textsc{Xenone100} detector, searching for signature of interactions arising from these models.

The second project is the development of a liquid xenon apparatus aiming at measuring quantum scintillation properties of liquid xenon, mainly searching for supperradiance effects. A better understanding of these processes can help in discriminating background, some of which is currently considered as irreducible.

The third project is on an external calibration technique for the \textsc{Xenon1T} detector. In order to discriminate between background and the expected signal, intensive calibration is required. Ton-scale detectors cannot be calibrated using the techniques previously used due to the self shielding of the xenon. I present a technique used in \textsc{Xenon1T} to obtain calibration of electromagnetic interactions with collimated external sources. 



\end{SingleSpace}
\clearpage