%
% File: Direxeno.tex
% Author: Ran Itay
% Description: Direxeno experiment
%
\let\textcircled=\pgftextcircled
\chapter{Direxeno}
\label{chap:Direxeno}
\initial{A}

%The relevance of collective spontaneous emission to direct DM searches, stems both from the
%directionality of the emitted radiation, and from its fast temporal variation. In LXe experiments,
%a cloud of excited diatomic molecules is produced by a recoiling nucleus or electron, where as
%discussed in the previous section, the cloud was estimated to have a complex topology with a typical
%size of O(100 nm). A fast temporal variation of the scintillation pulse for NRs and/or ERs may
%improve current discrimination ability between the two. A more powerful discrimination tool is
%the directionality of superradiant emission, which depends on the geometrical configuration of the
%initially excited radiators. If the produced cloud of excited molecules is oriented in a direction
%correlated with the primary recoil direction, the emitted scintillation light may be used for a 3D
%reconstruction of the recoil direction. For instance, an elongated cloud aligned in the recoil direction
%will radiate most of its energy into a solid angle around it.