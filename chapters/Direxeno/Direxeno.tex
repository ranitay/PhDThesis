%
% File: Direxeno.tex
% Author: Ran Itay
% Description: Direxeno experiment
%
\let\textcircled=\pgftextcircled
\chapter{Direxeno}
\label{chap:Direxeno}
\initial{I}n Liquid Xenon experiments, the cloud of excimers produced when a particle recoils energy in xenon is estimated to have a complex shape. Nonetheless it is expected to have a typical size of O(100nm). Hence it is expected that the excimer cloud might undergo a \superradiance emission. A variation in the temporal time of radiation between ERs and NRs might exist due to the size of the excimer cloud. This can improve the discrimination between background(ER) and signal(NR) for LXe DM detectors. Moreover weather the emitted radiation is correlated to the incoming exciting particle momentum, can be a more powerful tool for background reduction. Discarding events coming from the direction of the sun for example, is necessary once the neutrino floor(TODO add cite neutrino floor) will be crossed.  

Early studies conducted by Basov~\citep{BasovSRTheory} (experimentally) and in NIST(TODO add cite NIST) (theoretically) both present the option of generating a coherent radiation in the VUV regime (~178nm). The experimental setup was designed to bombard the LXe with 800~\,keV electron current pulse exciting (10\,ns) the LXe. The constant electron current causes a reverse population constantly. In contrary in direct detection of DM experiments, there is no "pump" producing this inverse population. The study of weather a cloud of excimers caused by a single interaction can exhibit \superradiance is still absence.

In this chapter we discuss Direxeno (Directional Xenon). An experimental setup designed at measuring \superradiance or any other non-linear effect in LXe.   
