%
% file: localoperator.tex
% author: Victor Brena
% description: Briefly describes properties of the local operator.
%

\chapter{EFT SIGNAL MODEL DETECTOR RESPONSE TABLE}
\label{app:response_table}

\initial{I}n this Appendix digital tables which can be used to construct an accurate signal model for the EFT analysis given any input recoil spectrum $\mathrm{d}R/\mathrm{d}E$ arising from a theoretical model is described. A visualization of the tables is shown in Fig.~\ref{fig:smeartable_highE}.

The signal model for the high-energy analysis region can be expressed analytically in the form
%
\begin{align}
\label{eq:high2D}
  \frac{\mathrm{d} R}{\mathrm{d}\cSi} &= \int \! \frac{\mathrm{d}R}{\mathrm{d}E} \cdot \epsilon_\mathrm{S1}(\cSi) \cdot \epsilon_\mathrm{S2'}(E) \cdot p_\mathrm{S1}(\mathrm{\cSi}|E) \, \mathrm{d}E \\
  &= \int \! \frac{\mathrm{d}R}{\mathrm{d}E} G(\cSi,E) \, \mathrm{d}E,
\end{align}
%
where $\epsilon_\mathrm{S1}(\cSi)$ and $\epsilon_\mathrm{S2'}(E)$ represent analysis cut efficiencies, $p_\mathrm{S1}(\mathrm{\cSi}|E)$ encodes detector effects, and $\mathrm{d}R/\mathrm{d}E$ gives the theoretically predicted nuclear recoil rate from WIMP scattering. In the second line, we emphasize that all the detector and analysis effects can be encoded in a single function $G(\cSi,E)$. To make a signal prediction for the bins in our analysis, this expression needs to be integrated over the appropriate range of $\cSi$ for each bin (and divided by 2 to account for the banding structure in $\cSiib$)
%
\begin{equation}
  R_\mathrm{bin_i} = \frac{1}{2}\int_{\mathrm{lower}_i}^{\mathrm{upper}_i} \! \frac{\mathrm{d} R}{\mathrm{d}\cSi} \, \mathrm{d}\cSi,
\end{equation}
%
With some simple rearrangement this rate can be written in terms of an integral over the detector response function $G$ as follows,
%
\begin{align}
  R_\mathrm{bin_i} &= \frac{1}{2}\int\frac{\mathrm{d} R}{\mathrm{d}E}\int_{\mathrm{lower}_i}^{\mathrm{upper}_i} \! G(\cSi,E) \, \mathrm{d}\cSi \, \mathrm{d}E \\
 &= \int\frac{\mathrm{d} R}{\mathrm{d}E} G'_i(E) \mathrm{d}E,
\end{align}
%
where in the last line we absorb the factor of $1/2$ into the definition of $G'_i$. We see here that the signal rate for each bin can be expressed as an integral over the recoil spectrum times a detector response function $G'_i$ for that bin. It is these detector response functions which are shown in Fig.~\ref{fig:smeartable_highE}, and which we provide digitally for use by the community. A low-resolution example is given in Table \ref{tab:smeartable_highE}. With these tables it is simple to produce a signal model for our analysis for any theoretical recoil spectrum. The functions $G'_i$ are provided for three values of the nuisance variable $\Leff$, namely the median value and values at $\pm 1 \sigma$ in $\Leff$. From these, along with the measured background rates given in Table.~\ref{table:BinDef}, one may construct a likelihood which accounts for uncertainties in $\Leff$. Alternatively simply using the $-1\sigma$ value produces quite an accurate prediction and is generally conservative.

\begin{figure}[c]
\centerline{\includegraphics[width=0.8\linewidth]{smeartable_highE}}
\mycaption[A visualization of the detector response table for $-1\sigma$ (i.e. conservative) $\Leff$]{A visualization of the detector response table for $-1\sigma$ (i.e. conservative) $\Leff$, as provided in the supplementary material. The y axis indicates the bins used for the high-energy signal region of this analysis (explained in ~\ref{table:BinDef}). The $x$ axis shows recoil energies, and the colors give the probability density for a recoil of a given recoil energy to produce an event in each analysis bin. To produce a signal model for this analysis, one simply multiplies the table values by $\mathrm{d}R/\mathrm{d}E$ and integrates over $E$. The result is the predicted signal rate for each analysis bin.}
\label{fig:smeartable_highE}
\end{figure}  

\begin{table}
{
\centerline{	\includegraphics[width=0.8\linewidth]{smearTable.png}}
}
\mycaption[Detector response table using $\Leff$ with constrained scaling parameter set to $-1\sigma$ value]{Detector response table using $\Leff$ with constrained scaling parameter set to $-1\sigma$ value. First column gives recoil energies, subsequent columns give the values of $G'_i(E)$ for each of the 9 high-energy analysis bins. The sampling is in steps of $10~\keVr$, which is too coarse to give an accurate signal model for very low WIMP masses, but is suitable for the mass range most relevant to our analysis. Higher resolution $G'_i(E)$ functions, and $G'_i(E)$ functions for other values of $\Leff$, are given in supplementary material. 
\label{tab:smeartable_highE}
}
\end{table}  

